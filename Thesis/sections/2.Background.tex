% !TEX root = ..\main.tex
\chapter{Background}\label{ch:Background}
In this chapter we present the background information needed for this thesis. In addition, we also define some terminology that is going to be used in throughout the thesis.

\section{Terminology}

\reminder{This is only to remember the terms from the literature}
Terms that I have to decide which to use and how to define them:

Library / Package (kikas2017)

Library version / Package version

Application (kikas2017)

Project (pashchenko2018)

Product (FASTEN)

Repositories (kikas2017)

Dependency

Direct dependency (kikas2017, pashchenko2018)

Transitive dependency (kikas2017, pashchenko2018)

Halted dependency (pashchenko2018)

Bloated dependency (soto2020comprehensive)

Deployed / non-deployed dependency (pashchenko2018)

Dependency network (kikas2017)

Ecosystem (kikas2017)

\reminder{}

\section{Dependency management}

\section{Dependency Networks}
\subsection{Package Dependency Network (PDN)}
\subsection{Call Dependency Network (CDN)}

\section{Coupling}
When assessing the quality of software, there are many aspects that are considered and measured. One of these, in particular for Object-Oriented systems, is coupling. Coupling measures the degree of dependency between two different parts of a system. In the literature of coupling metrics, these have been used to measure the dependency of the elements of the same system, or to give an overview of the coupling within a system. However, in this thesis we measure the coupling, or degree of dependency, between two different systems.

As described in section \ref{section:scope}, this thesis is focused in the structural metrics, based on static source code analysis.

There are many structural coupling metrics, each one measuring a different type of coupling from a different perspective, depending on the purpose for which the metrics are needed. In order to define type of metric needed to measure the dependency between products, we have used the framework defined by Briand et al. \cite{briand1999unified}, which unifies the frameworks defined by Eder et al. \cite{eder1994coupling}, Hitz and Montazeri \cite{hitz1995measuring}, and Briand et al. \cite{briand1997investigation}.

According to the unified framework defined by Briand et al., the coupling metrics have certain characteristics defining which type of coupling are they measuring. In particular, six criteria are defined:

\begin{itemize}
  \item \textbf{Type of connection:} This criteria defines what creates the coupling, which type of dependency is measured, how the two elements are connected.

  \item \textbf{Locus of impact:} If the coupling is import or export. In other words, if the class for which coupling is being measured is the client or the server of the relationship.

  \item \textbf{Granularity of the measure:} The domain at which coupling is measured (e.g. class-level). The granularity also refers to how the metric counts the connection (e.g. binary evaluation, or counting each one of the connections).

  \item \textbf{Stability of the server:} According to Briand et al. one of the applications of this criteria could be to distinguish between dependency with standard libraries and others. However, it is not used by the existing metrics.

  \item \textbf{Direct and indirect coupling:} Whether the connection between the two elements is direct, or transitive (there is at least one other element connecting the two). The metrics that do not account for indirect coupling, can be adapted by calculating the transitive closure of the metric.

  \item \textbf{Inheritance:} In this criteria, Briand et al. group the position of the metric respecting special cases such as inheritance and polymorphism.
\end{itemize}

\section{Effort measurement}
