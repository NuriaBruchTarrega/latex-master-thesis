% !TEX root = ..\main.tex
\chapter{Background}\label{ch:Background}
In this chapter we present the background information needed for this thesis. In addition, we also define some terminology that is going to be used in throughout the thesis.

\section{Terminology}

\reminder{This is only to remember the terms from the literature}
Terms that I have to decide which to use and how to define them:

Library / Package (kikas2017)

Library version / Package version

Application (kikas2017)

Project (pashchenko2018)

Product (FASTEN)

Repositories (kikas2017)

Dependency

Direct dependency (kikas2017, pashchenko2018)

Transitive dependency (kikas2017, pashchenko2018)

Halted dependency (pashchenko2018): Dependency in which the server library is no longer updated. This means that if a vulnerability is detected in this library, it will not be fixed.

Bloated dependency (soto2020comprehensive): Direct or transitive dependency that is included when compiling the application, but it is not really used by the code of the application.

Deployed / non-deployed dependency (pashchenko2018)

Dependency network (kikas2017)

Ecosystem (kikas2017)

\reminder{}

\section{Dependency management}

\section{Dependency Networks}
\subsection{Package Dependency Network (PDN)}
\subsection{Call Dependency Network (CDN)}

\section{Coupling}
When assessing the quality of software, there are many aspects that are considered and measured. One of these, in particular for Object-Oriented systems, is coupling. Coupling measures the degree of dependency between two different parts of a system. In the literature of coupling metrics, these have been used to measure the dependency of the elements of the same system, or to give an overview of the coupling within a system. However, in this thesis we measure the coupling, or degree of dependency, between two different systems.

As described in section \ref{section:scope}, this thesis is focused in the structural metrics, based on static source code analysis.

There are many structural coupling metrics, each one measuring a different type of coupling from a different perspective, depending on the purpose for which the metrics are needed. In order to define type of metric needed to measure the dependency between products, we have used the framework defined by Briand et al. \cite{briand1999unified}, which unifies the frameworks defined by Eder et al. \cite{eder1994coupling}, Hitz and Montazeri \cite{hitz1995measuring}, and Briand et al. \cite{briand1997investigation}.

According to the unified framework defined by Briand et al., the coupling metrics have certain characteristics defining which type of coupling are they measuring. In particular, six criteria are defined:

\begin{itemize}
  \item \textbf{Type of connection:} This criteria defines which mechanism creates coupling, which type of dependency is measured, how the two elements are connected. The different types of connection, as described by Briand et al. \cite{briand1999unified} can be found in Table \ref{table:types-connections}.

  \item \textbf{Locus of impact:} If the coupling is import or export. In other words, if the class for which coupling is being measured is the client or the server of the relationship.

  \item \textbf{Granularity of the measure:} The detail at which the metric calculates coupling. It is defined by 1) The domain at which coupling is measured (e.g. class-level) and 2) how the metric counts the connections (e.g. binary evaluation or counting each one of the connections). The six options to count connections as defined by Briand et al. \cite{briand199unified}, can be found in Table \ref{table:counting-connections}.

  \item \textbf{Stability of the server:} In the framework by Briand et al. the servers are classified as unstable if these are  "subject to development or modification in the project at hand", and stable if these "are not subject to change in the project at hand". The last one includes classes imported from libraries. According to Briand et al., the coupling with an unstable class is represents more risk than coupling with a stable class. However, the studied metrics in the framework do not make use of this criteria, and treat all classes with the same importance.

  \item \textbf{Direct and indirect coupling:} Whether the connection between the two elements is direct, or transitive (there is at least one other element connecting the two). The metrics that do not account for indirect coupling, can be adapted by calculating the transitive closure of the metric.

  \item \textbf{Inheritance:} In this criteria, Briand et al. group the position of the metric respecting special cases such as inheritance and polymorphism.
\end{itemize}

\begin{table}[ht!]
    \begin{center}
    \begin{tabularx}{\textwidth}{|l|l|l|X|}
    \hline
    \# & Client Item & Server Item & Description \\
    \hline\hline
    1   & attribute \textit{a} of a class \textit{c} & class \textit{d}, d != c & class \textit{d} is the type of \textit{a} \\
    \hline
    2   & method \textit{m} of a class \textit{c} & class \textit{d}, d != c  & class \textit{d} is the type of a parameter of \textit{m}, or the return type of \textit{m} \\
    \hline
    3   & method \textit{m} of a class \textit{c} & class \textit{d}, d != c  & class \textit{d} is the type of a local variable of \textit{m} \\
    \hline
    4   & method \textit{m} of a class \textit{c} & class \textit{d}, d != c  & class \textit{d} is the type of a parameter of a method invoked by \textit{m} \\
    \hline
    5   & method \textit{m} of a class \textit{c} & \begin{tabular}[c]{@{}l@{}}attribute \textit{a} of a\\ class \textit{d}, d != c \end{tabular}  & \textit{m} references \textit{a} \\
    \hline
    6   & method \textit{m} of a class \textit{c} & \begin{tabular}[c]{@{}l@{}}method \textit{m'} of a\\ class \textit{d}, d != c \end{tabular} & \textit{m} invokes \textit{m'} \\
    \hline
    7   & class \textit{c} & class \textit{d}, d != c  & high-level relationships between classes, such as \textit{uses} or \textit{consists-of} \\
    \hline
    \end{tabularx}
    \end{center}
    \caption{Types of connections, obtained from \cite{briand1999unified}}
    \label{table:types-connections}
\end{table}

\begin{table}[htb!]
    \begin{center}
    \begin{tabularx}{\textwidth}{|l|l|X|}
    \hline
    \begin{tabular}[c]{@{}l@{}}Counting\\ connections\end{tabular} & Level & Description \\
    \hline\hline
    A   & \begin{tabular}[c]{@{}l@{}}Method or\\ attribute\end{tabular} & count individual connections  \\
     \hline
    B   & \begin{tabular}[c]{@{}l@{}}Method or\\ attribute\end{tabular} & count the number of distinct items at the other end of the connections  \\
     \hline
    C   & Class & add up the number of connections counted as in A) for each method or attribute of the class   \\
     \hline
    D   & Class & add up the number of connections counted as in B) for each method or attribute of the class   \\
     \hline
    E   & Class & count the number of distinct items at the end of connections starting from or ending in methods or attributes of the class    \\
     \hline
    F   & Class & for a class c, count the number of other classes to which there is at least  one connection  \\
    \hline
    \end{tabularx}
    \end{center}
    \caption{Counting connections, obtained from \cite{briand1999unified}}
    \label{table:counting-connections}
\end{table}

\section{Effort measurement}
