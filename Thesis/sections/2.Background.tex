% !TEX root = ../main.tex
\chapter{Background}\label{ch:Background}
In this chapter, we present the background information needed for this thesis. In addition, we also define some terminology that is going to be used throughout the thesis.

\section{Terminology}
In this section, we will review the terms used in some of the related literature, particularly those papers that specify the terminology used. The different terms used are compared by explaining the differences in both the words and the meaning. Finally, we will specify the terms used in this thesis and its definition.

\begin{table}[htb!]
    \begin{center}
    \begin{tabular}{|l|c|c|c|c|}
    \hline
    Term & \cite{pashchenko2018vulnerable} & \cite{kikas2017structure} & \cite{fasten2019survey} & \cite{soto2020comprehensive} \\
    \hline
    Library                           & x &   & x & x \\
    Package                           &   & x & x &   \\
    Application                       &   & x &   &   \\
    Project                           & x & x &   & x \\
    \hline
    Version                           &   & x & x & x \\
    Instance                          & x &   &   &   \\
    Release                           &   &   & x &   \\
    Artifact                          &   &   &   & x \\
    \hline
    Package manager                   &   &   & x & x \\
    Package repository                &   &   & x &   \\
    \hline
    Dependency                        & x & x & x & x \\
    Reverse dependency                &   & x &   &   \\
    Inherited dependency              &   &   &   & x \\
    Direct/Transitive dependency      & x & x & x & x \\
    Direct/Indirect dependency        &   &   & x &   \\
    Deployed/non-deployed dependency  & x &   & x &   \\
    Own/third-party dependency        & x &   &   &   \\
    Halted dependency                 & x &   & x &   \\
    Bloated dependency                &   &   &   & x \\
    \hline
    Dependency tree                   & x &   &   & x \\
    Dependency network                &   & x & x &   \\
    Ecosystem                         &   & x &   &   \\
    \hline
    \end{tabular}
    \end{center}
    \caption{Terminology from the literature}
    \label{table:terminology}
\end{table}

\blankls
First, we are going to discuss the differences between the first four terms of the table. A library is a software unit, which is distributed separately and can be reused in other software components. The term library is, in some cases, replaced by the term package. The term application is used to differentiate the software components that are available in repositories to be reused (libraries) and those that are not (applications) \cite{kikas2017structure}.
The term project is used in different papers with different meanings. Pashchenko et al. \cite{pashchenko2018vulnerable} use it to refer to a group of libraries developed or maintained by the same team of developers. Kikas et al. \cite{kikas2017structure} use it to refer to both library and application. Finally, Soto-Valero et al. use it to refer to Maven Projects.

Each library, available in a package repository, can have different versions. Apart from the word version, the literature also uses the terms instance and release, but with the same meaning. The word artifact is used by Soto-Valero et al. to refer to a Maven Artifact, which corresponds to a Maven Project version.

The papers that specifically define package manager or package repository in the terminology, it is because these tools have an important role in the research.

When describing the relationship in which a library uses another, relevant literature uses the term dependency. However, different types of dependencies are defined depending on the paper and the research done.

The dependencies declared in a library are always called direct dependencies, and those introduced by the direct dependencies are called transitive or indirect.

There is also the distinction between deployed and non-deployed. The deployed ones are those included in the software product once it is in production, and the non-deployed ones are those used for development tasks (e.g., testing libraries).

When considering the risk associated with a dependency, it is important to consider whether the library is own or third-party. An own dependency is maintained by the same team that develops the product that has the dependency.

Another particular case when considering risk is halted dependencies. These dependencies are not updated anymore, which means that if a vulnerability is found, it will not be fixed.

The last type of dependency is the bloated dependency. These dependencies can be direct or not, but have the characteristic that the library is never used and only increase the dependency tree's size without the need.

Finally, the terms dependency tree and dependency network refer to the graph created by the dependencies. In these graphs, the library versions are the nodes, and the dependencies are the edges. The ecosystem includes all the libraries and applications involved in the dependency network.

\blankl
Based on the terminology discussed above, we define the terms that are used within this thesis.

\begin{itemize}
  \item \textbf{Library:} A software artifact that is distributed individually. It can have different implementations distinguished by versions. The model created in this thesis is meant to analyze the dependencies of any software product. However, for simplicity during the thesis, we always to libraries.

  \item \textbf{Version:} A version of a library that contains an implementation of the library. Each version has specific metadata associated to build the version successfully. Among other data, it specifies which versions of other libraries it is using.

  \item \textbf{Dependency:} When a library version uses a version of another library, it creates a relationship between the versions of the two libraries, a dependency. In particular, the first library version depends on the second one.

  \item \textbf{Direct and transitive dependencies:} A direct dependency is when a library version directly uses a version of another library. A transitive dependency is given when a library version uses another one indirectly, through other library versions that it depends on.

  \item \textbf{Unused dependencies:} We have decided to use the term unused dependency, instead of bloated dependency since we find it easier to understand. A dependency is unused when included in the library version's dependency set, but the library never reaches it. This could happen with both direct and transitive dependencies for different reasons.

  \item \textbf{Dependency network:} Graph that represents the dependencies between library versions. In a dependency network, the library versions are the nodes, and the dependencies between them are the edges.

  \item \textbf{Ecosystem:} A set of libraries that have versions with dependencies. When the libraries are updated (new versions), the ecosystem evolves.
\end{itemize}

\section{Dependency management}
There exist several dependency managers. Tn this thesis, we will be analyzing Java libraries, and in particular, we will use projects existing in the Maven ecosystem, \textit{Maven Central Repository}\footnote{\url{https://repo1.maven.org/maven2/}}. Therefore, the dependency manager used is the one included in Maven. This section contains the description of different characteristics of Maven, including the different types of dependencies and how these are declared.

\subsection{Maven Dependencies}
The configuration of a Maven Project, including the dependency management, is done in the \textit{Project Object Model (POM)} file. Apart from defining the dependencies of a project, the POM file contains the project's description and the build plugins that it uses.

The following parameters define a project:

\begin{itemize}
  \item \textbf{GroupID:} The identifier of the group or company that developed the project.
  \item \textbf{ArtifactID:} The identifier of the project itself.
  \item \textbf{Version:} The version of the implementation of the project.
  \item \textbf{Packaging:} The packaging method that the project uses. Although other packaging methods are available, \textit{jar} files are the default ones and can be used to analyze the bytecode of the libraries.
\end{itemize}

\paragraph{Module hierarchy}
A Maven Project can be configured using two different strategies. First, as a single module, it will only have one \textit{POM} file, and that only one packaging will result in the build of the project. The second option is to create a multi-module project. In this case, the project has multiple \textit{POM} files. These \textit{POM} files can have a defined hierarchy in which there is a parent \textit{POM} that has children \textit{POMs}, these children will \textit{inherit} dependencies from the parent file. For the development of the PoC of this thesis, a module with a \textit{POM} file, even if it has a parent module, is considered a library, since it can be used in a different library as an individual dependency.

\paragraph{Dependencies and DependencyManagement}
There are two sections in a \textit{POM} file that are used for dependency management purposes: \texttt{dependencies} and \texttt{dependencyManagement}.

The \texttt{dependencyManagement} section is used in multi-module projects. It is used to define certain dependency information (e.g., the version of the artifacts). It is used in the parent \textit{POM} to simplify the dependency definition of the children's files.

In the \texttt{dependencies} section is where the dependencies are declared. If a parent file has dependencies declared in this section, these will always be inherited by the children's files.

Maven uses both sections of the \textit{POM} file to resolve the dependencies of a library.

\paragraph{Scope of the dependencies}
One of the main mechanisms that the Maven dependency manager offers is the dependency scope specified for each dependency included in a \textit{POM} file. A direct dependency's scope affects how the transitive dependencies are treated, except the scope import. There are 6 different scopes:

\begin{itemize}
  \item \textbf{Compile:} The default scope, all direct dependencies without a specified scope, have compiled scope. These dependencies are available in the library's classpath and will be propagated as transitive dependencies to the libraries that depend on the current library.

  \item \textbf{Provided:} This scope type means that the dependency is expected to be provided by the JDK or the container during runtime. The dependencies with scope provided do not propagate transitive dependencies and are only available in the classpath on compilation and test.

  \item \textbf{Runtime:} In this case, the dependency is only needed during the execution, and therefore not necessary during compilation. It is available in the classpath during runtime and test. Runtime dependencies are propagated as transitive dependencies.

  \item \textbf{Test:} This scope specifies a dependency that is only used for testing purposes. It is available in the classpath during test and execution phases. Test dependencies do not propagate as transitive dependencies.

  \item \textbf{System:} Similar to \textit{provided}, but it is necessary to indicate the path to the jar of the dependency. It may cause problems if the product is built in a machine where the indicated path does not match the actual one. This scope is not transitive.

  \item \textbf{Import:} This scope is only available for dependencies declared in the section \textit{DependencyManagement} and with specified type \textit{pom}. It indicates that the dependency should be replaced with the dependencies declared in its pom. Therefore, these dependencies are replaced and do not affect transitivity.
\end{itemize}

\paragraph{Optional dependencies and exclusions}
In Maven, it is possible to declare dependencies as \textit{optional}. The primary use case for this is when a project is not divided into sub-modules, and specific dependencies are only used in certain parts of the project. In this case, it might be possible to use the project without using these parts, and the dependency might not be necessary. Declaring the dependency as optional allows saving both space and memory when the dependency is not used.

Another feature of the dependency management available in Maven is the dependency \textit{exclusions}. This feature has the goal of saving part of the memory and space used by transitive dependencies. When a particular transitive dependency is not used in your project, it is possible to specify the transitive dependency exclusion. It could be useful if you use only a part of a direct dependency that does not need the transitive dependency. The excluded dependency will not be included in the dependency tree of the project and will not be imported in the library's classpath.

\paragraph{Dependency resolution}
To resolve the dependencies, Maven uses the \textit{POM} file and scopes recursively to create the dependency tree. However, another aspect of dependency resolution is the version of each of the dependencies. It is possible that a dependency tree contains the same dependency more than once, and maybe with different versions. Maven uses the dependency mediation algorithm to resolve the version of the dependencies. The algorithm's strategy is the "nearest definition," which consists of choosing the version of the dependency that is closer to the root in the dependency tree. If one artifact is declared twice with different versions and at the same level, the dependency mediation will choose the first declaration of the dependency.

\section{Coupling}\label{section:bg-coupling}
When assessing the quality of software, many aspects are considered and measured. One of these, in particular for Object-Oriented systems, is coupling. Coupling measures the degree of dependency between two different parts of a system. In the literature of coupling metrics, these have been used to measure the dependence of the same system elements or give an overview of the coupling within a system. However, in this thesis, we measure the coupling, or degree of dependency, between two systems.

Therefore, we propose re-using the already existing coupling metrics, meant to measure the coupling between units of the \textit{same} project and adapt them to measure coupling \textit{between} projects.

\blankl
According to Poshyvanyk and Marcus in \cite{poshyvanyk2006conceptual}, there are six main groups of coupling metrics:

\begin{itemize}
  \item \textbf{Structural coupling metrics:} Measured directly from static source code analysis. Largely studied by the literature about coupling \cite{briand1999unified, poshyvanyk2006conceptual, briand1997investigation, eder1994coupling}.

  \item \textbf{Dynamic coupling measures:} Measured using dynamic code analysis. \textit{"Introduced as the refinement to existing coupling measures due to gaps in addressing polymorphism, dynamic binding, and the presence of unused code by static structural coupling measures"} \cite{poshyvanyk2006conceptual}.

  \item \textbf{Evolutionary and Logical coupling:} According to Zimmermann and Diehl \cite{zimmermann2005mining}, the evolutionary coupling can: \textit{"tell us which parts of the system are coupled by common changes or cochanges."}

  \item \textbf{Coupling measures based on information entropy approach:} Coupling metrics based on the information-theory approach, such as the metrics proposed by Allen and Khoshgoftaar in \cite{allen1999measuring}.

  \item \textbf{Conceptual coupling metrics:} Based on the semantic similarity between the elements. This is the focus of the work from Poshyvanyk and Marcus \cite{poshyvanyk2006conceptual}.

  \item \textbf{Coupling metrics for specific types of software applications:} Specialized coupling metrics for certain kinds of projects, such as knowledge-based systems or aspect-oriented approach.
\end{itemize}

\blankl
Since this research is aimed to be independent of the domain, the last category is not considered in this thesis. Moreover, the evolutionary coupling is not possible to be applied in our context. Likely, the separate projects are not going to evolve simultaneously, given that the same team does not develop them. Finally, the research of this project, owing to the time limitation, will be centered on the structural metrics.

These metrics are going to be proposed as a first step to measure the degree of library dependency. Nevertheless, the metrics can be extended and calculated more accurately by adding dynamic coupling and information entropy approach metrics in future work.

There are many structural coupling metrics, each measuring a different type of coupling from a different perspective, depending on the purpose for which the metrics are needed. To define the necessary metrics to measure the dependency between products, we have used the framework described by Briand et al. \cite{briand1999unified}, which unifies the frameworks defined by Eder et al. \cite{eder1994coupling}, Hitz and Montazeri \cite{hitz1995measuring}, and Briand et al. \cite{briand1997investigation}.

According to the unified framework defined by Briand et al. \cite{briand1999unified}, the coupling metrics have specific characteristics defining which type of coupling they are measuring. In particular, six criteria are defined:

\begin{itemize}
  \item \textbf{Type of connection:} This criterion defines which mechanism creates coupling, which type of dependency is measured, how the two elements are connected. The different types of connection, as described by Briand et al. \cite{briand1999unified} can be found in Table \ref{table:types-connections}.

  \item \textbf{Locus of impact:} If the coupling is import or export. In other words, if the class for which coupling is being measured is the client or the server of the relationship.

  \item \textbf{Granularity of the measure:} The detail at which the metric calculates coupling. It is defined by 1) The domain at which coupling is measured (e.g., class-level) and 2) how the metric counts the connections (e.g., binary evaluation or counting each one of the connections). The six options to count connections as defined by Briand et al. \cite{briand1999unified}, can be found in Table \ref{table:counting-connections}.

  \item \textbf{Stability of the server:} In the framework by Briand et al. \cite{briand1999unified}, the servers are classified as unstable if these are  "subject to development or modification in the project at hand" and stable if these "are not subject to change in the project at hand." The last one includes classes imported from libraries. According to Briand et al., coupling with an unstable class represents more risk than coupling with a stable class. However, the framework's studied metrics do not make use of this criterion and treat all classes with the same importance.

  \item \textbf{Direct and indirect coupling:} Does the connection between the two elements are direct or transitive (there is at least one other element connecting the two). The metrics that do not account for indirect coupling can be adapted by calculating the metric's transitive closure.

  \item \textbf{Inheritance:} In this criteria, Briand et al. \cite{briand1999unified} define the position of the metric respecting exceptional cases such as inheritance and polymorphism.
\end{itemize}

\begin{table}[ht!]
    \begin{center}
    \begin{tabularx}{\textwidth}{|l|l|l|X|}
    \hline
    \# & Client Item & Server Item & Description \\
    \hline\hline
    1   & attribute \textit{a} of a class \textit{c} & class \textit{d}, d != c & class \textit{d} is the type of \textit{a} \\
    \hline
    2   & method \textit{m} of a class \textit{c} & class \textit{d}, d != c  & class \textit{d} is the type of a parameter of \textit{m}, or the return type of \textit{m} \\
    \hline
    3   & method \textit{m} of a class \textit{c} & class \textit{d}, d != c  & class \textit{d} is the type of a local variable of \textit{m} \\
    \hline
    4   & method \textit{m} of a class \textit{c} & class \textit{d}, d != c  & class \textit{d} is the type of a parameter of a method invoked by \textit{m} \\
    \hline
    5   & method \textit{m} of a class \textit{c} & \begin{tabular}[c]{@{}l@{}}attribute \textit{a} of a\\ class \textit{d}, d != c \end{tabular}  & \textit{m} references \textit{a} \\
    \hline
    6   & method \textit{m} of a class \textit{c} & \begin{tabular}[c]{@{}l@{}}method \textit{m'} of a\\ class \textit{d}, d != c \end{tabular} & \textit{m} invokes \textit{m'} \\
    \hline
    7   & class \textit{c} & class \textit{d}, d != c  & high-level relationships between classes, such as \textit{uses} or \textit{consists-of} \\
    \hline
    \end{tabularx}
    \end{center}
    \caption{Types of connections, obtained from \cite{briand1999unified}}
    \label{table:types-connections}
\end{table}

\begin{table}[htb!]
    \begin{center}
    \begin{tabularx}{\textwidth}{|l|l|X|}
    \hline
    \begin{tabular}[c]{@{}l@{}}Counting\\ connections\end{tabular} & Level & Description \\
    \hline\hline
    A   & \begin{tabular}[c]{@{}l@{}}Method or\\ attribute\end{tabular} & count individual connections  \\
     \hline
    B   & \begin{tabular}[c]{@{}l@{}}Method or\\ attribute\end{tabular} & count the number of distinct items at the other end of the connections  \\
     \hline
    C   & Class & add up the number of connections counted as in A) for each method or attribute of the class   \\
     \hline
    D   & Class & add up the number of connections counted as in B) for each method or attribute of the class   \\
     \hline
    E   & Class & count the number of distinct items at the end of connections starting from or ending in methods or attributes of the class    \\
     \hline
    F   & Class & for a class c, count the number of other classes to which there is at least  one connection  \\
    \hline
    \end{tabularx}
    \end{center}
    \caption{Counting connections, obtained from \cite{briand1999unified}}
    \label{table:counting-connections}
\end{table}

\begin{table}[p]
    \begin{center}
    \begin{tabular}{|l|l|l|l|l|l|l|}
    \hline
    \rot{Metric} & \rot{Inheritance} & \rot{Locus of impact} & \rot{Types of connection} & \rot{Domain of measure} & \rot{Counting connections   } & \rot{Indirect coupling} \\ \hline \hline
    CBO           & both  & both    & 5, 6  & class               & F     & no      \\
    CBO'          & no    & both    & 5, 6  & class               & F     & no      \\
    \hline
    $RFC_\alpha$  & both  & import  & 6     & class               & E     & depends \\
    RFC           & both  & import  & 6     & class               & E     & no      \\
    RFC'          & both  & import  & 6     & class               & E     & yes     \\
    \hline
    MPC           & both  & import  & 6     & class               & C     & no      \\
    \hline
    DAC           & both  & import  & 1     & class               & C     & no      \\
    DAC'          & both  & import  & 1     & class               & D     & no      \\
    \hline
    COF           & no    & both    & 5, 6  & system              & F     & no      \\
    \hline
    ICP           & both  & import  & 6     & method, class, set  & A, C  & no      \\
    IH-ICP        & only  & import  & 6     & method, class, set  & A, C  & no      \\
    NIH-ICP       & no    & import  & 6     & method, class, set  & A, C  & no      \\
    \hline
    IFCAIC        & no    & import  & 1     & class               & C     & no      \\
    ACAIC         & only  & import  & 1     & class               & C     & no      \\
    OCAIC         & no    & import  & 1     & class               & C     & no      \\
    FCAEC         & no    & export  & 1     & class               & C     & no      \\
    DCAEC         & only  & export  & 1     & class               & C     & no      \\
    OCAEC         & no    & export  & 1     & class               & C     & no      \\
    \hline
    IFCMIC        & no    & import  & 2     & class               & C     & no      \\
    ACMIC         & only  & import  & 2     & class               & C     & no      \\
    OCMIC         & no    & import  & 2     & class               & C     & no      \\
    FCMEC         & no    & export  & 6     & class               & C     & no      \\
    DCMEC         & only  & export  & 6     & class               & C     & no      \\
    OCMEC         & no    & export  & 6     & class               & C     & no      \\
    \hline
    OMMIC         & no    & import  & 6     & class               & C     & no      \\
    IFMMIC        & no    & import  & 6     & class               & C     & no      \\
    AMMIC         & only  & import  & 6     & class               & C     & no      \\
    OMMEC         & no    & export  & 6     & class               & C     & no      \\
    FMMEC         & no    & export  & 6     & class               & C     & no      \\
    DMMEC         & only  & export  & 6     & class               & C     & no      \\
    \hline
    \end{tabular}
    \end{center}
    \caption{Coupling metrics comparison}
    \label{table:coupling-metrics}
\end{table}


\blankl
Based on these criteria, Briand et al. classify the existing coupling metrics, according to their definitions \cite{briand1999unified}. The comparison of all the metrics, according to the criteria of the framework, can be found in Table \ref{table:coupling-metrics}. The criteria stability of the server has been excluded from the table since none of the metrics consider it.

\paragraph{Coupling Between Objects (CBO)} This metric counts the number of other classes to which the client class coupled. This metric has two definitions: the original definition, CBO' in Table \ref{table:coupling-metrics}, which does not count inheritance. Then, there is the revised definition of the metric \cite{chidamber1994metrics}, which does include inheritance.

\paragraph{Response for Class (RFC)} This metric calculates the response set of a class. According to Chidamber and Kemerer \cite{chidamber1994metrics}, \textit{"The response set of a class is a set of methods that can potentially be executed in response to a message received by an object of that class"}. The return set includes the methods called directly by the class, as well as the methods that are called by transitivity. In the framework by Briand et al. \cite{briand1999unified}, it is considered that the inherited methods should be included in this set since they can be executed to respond to a message received in the class. Based on this definition, there are three defined metrics, the first one being $RFC_\alpha$ \cite{churcher1995towards}. The $\alpha$ defines the number of nested levels of transitivity considered in the calculation of the metric. The other two metrics are particular cases of this one: RFC corresponds to when $\alpha = 1$, and RFC' when $\alpha = \infty$.

\paragraph{Message Passing Coupling (MPC)} This metric, created by Li and Henry \cite{li1993object}, counts invocations from the new methods of a class to methods of other classes. This means that the inherited methods are not considered, but it is unclear how it treats overridden methods or calls to inherited methods. Briand et al. \cite{briand1999unified}, to eliminate ambiguity, redefined the metric as \textit{"the number of static invocations of methods not implemented in c by methods implemented in c"}.

\paragraph{Data Abstraction Coupling (DAC)} This metric was also defined by Li and Henry \cite{li1993object} as follows: \textit{"number of ADTs defined in a class"}, where ADT is abstract data type. However, this definition does not specify how the metric should count the connections or consider inherited ADTs. Because of this ambiguity in the original definition of the metric, Briand et al. \cite{briand1999unified} redefined the metric: \textit{"DAC is the number of not inherited attributes that have a class as their type. The number of the classes used as types for attributes is counted by DAC'"}.

\paragraph{Coupling Factor (COF)} COF is the only metric of which the domain of measurement is the entire system, and was defined by Abreu et al. \cite{abreu1995toward}. COF calculates the number of relations between classes of the system, which are not related through inheritance. The relations are counted in a binary manner. The factor is normalized between 1 and 0 by dividing the number of relations by the system's maximum number of relations possible. This way, it is possible to compare systems of different sizes.

\paragraph{Information-flow-based Coupling (ICP)} The original ICP metric counts \textit{"for method m of class c, the number of polymorphically invoked methods of other classes, weighted by the number of parameters of the invoked method."} Sadly, we have not been able to obtain the original paper, but it is described by Briand et al. \cite{briand1999unified}. From this description, the metrics IH-ICP and NIH-ICP are defined. IH-ICP counts only inheritance-based coupling, whereas NIH-ICP counts coupling to those classes with no inheritance relationship. Finally, the metric ICP is the sum of the previous two.

\paragraph{Suite of metrics by Briand et al.} This set of metrics was defined by Briand et al. with their previous framework for coupling metrics \cite{briand1997investigation}. This framework was specially created for C++, and therefore, it has some extensions specific to this language.
The metrics of the set are named according to three criteria: relationship, locus, and type. Each metric's name is composed in the following way: the initials of the relationship, the initials of the type of interaction, and the initials of the locus. These initials are described below.

\blankl
There are three types of relationships, listed below, which can be used to determine the coupling of a class $c$. All the definitions have been obtained from \cite{briand1997investigation}.

\begin{itemize}
  \item Inheritance (A, D): Interactions from a class to its antecessors or descendants, depending on the locus.
  \item Friendship (F, IF): Extension for C++, interactions from class to all the classes declared as friends or the classes that declare it their friend (inverse friends), depending on the locus.
  \item Other (O): interaction with classes that do not have an inheritance or friendship relationship.
\end{itemize}

\blankls
The three different types of interaction described by Briand et al. \cite{briand1997investigation} are the following:
\begin{itemize}
    \item Class-Attribute (CA): "There is a class-attribute (CA-) interaction from class c to class d if an attribute of class c is of type class d."
    \item Class-Method (CM): "There is a class-method (CM-) interaction from class c to class d if a newly defined method of class c has a parameter of type class d."
    \item Method-Method (MM): "There is a method-method (MM-) interaction from class c to class d, if a method implemented at class c statically invokes a method of class d (newly defined or overriding), or receives a pointer to such a method."
\end{itemize}

\blankls
Finally, the two types of locus are:
\begin{itemize}
  \item Export from a class (EC): "Change flows away from a class" related to the descendants (D) and the friends (F).
  \item Import to a class (IC): "Change flows towards a class" related to the ancestors (A) and the inverse friends (IF).
\end{itemize}

\blankls
The suite of metrics is defined based on all possible combinations of these three criteria.

\section{Metrics validation}
This thesis includes evaluating and validating the metrics included in the proposed model by using the proof-of-concept. Since there is no unique way to validate metrics which is globally accepted and used, various approaches are adopted. In the paper \cite{srinivasan2014software}, Srinivasan et al. explain that there are two fundamental approaches for metric validation: \textit{theoretically} and \textit{empirically}. Therefore, to provide a check of validity that is as broad as possible, a mixture of these two approaches will be used during this project. However, this research does not contain a full validation of the metrics due to time constraints.

For the coupling metrics' theoretical validity, these are validated according to the \textit{Mathematical Properties of Measures for Coupling} \cite{srinivasan2014software}. Also, a subset of the aspects presented by Meenely et al. \cite{Meneely2012} are also used to validate all the metrics in the model; in particular, we focus on Actionability and Definition validity.
