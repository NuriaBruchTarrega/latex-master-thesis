% !TEX root = ..\main.tex
\chapter{Introduction}\label{ch:Introduction}

\section{Problem statement}

\subsection{Context}
At present, there are many open-source libraries available for all developers to reuse the features that these libraries implement. This practice is becoming more and more popular since it allows to reuse previously developed code, and helps developers avoid implementing the same functionalities multiple times.

When a developer uses an open-source library in a project, it creates a dependency between the application and the library. This implies that a significant number of projects depend on other open-source projects, and it adds the task of managing these dependencies to the maintenance tasks of the project. How to maintain the dependencies is not a trivial task, and it is one of the problems that the field of software engineering is trying to solve \cite{kula2014visualizing}.

The management and maintenance of the dependencies of a project is an important task. Open-source libraries, just like any other software project, can have security vulnerabilities that may affect the projects that depend on these libraries. For example, some vulnerabilities can become security problems that can have a negative impact in terms of integrity, privacy or availability.

\blankl
Currently, developers have at their disposal package managers, to ease the task of managing the dependencies of their projects. However, the dependency management available in these package managers only evaluate if a dependency exists or not and a more detailed risk evaluation is missing \cite{hejderup2018prazi}. There is no way to evaluate how much a project depends on a library.

Furthermore, it could happen that the developers of a project decide to replace one of the dependencies of the project with another one. This could happen in case a library has vulnerabilities or is deprecated, to prevent the vulnerabilities from affecting the project. However, replacing a dependency could be a costly process. It involves identifying which parts of the project are affected by the dependency, and which parts of the library are being used and need replacement.

\blankl
Therefore, this thesis has the aim to perform a more detailed evaluation of the dependencies. A set of metrics is proposed to measure the dependencies between projects and the open-source libraries these use. In addition, we want to propose a way to measure the effort that would be required to replace a dependency with a new one. We are going to define a method to estimate effort considering which parts of the code are affected by the dependency.

\subsection{Host organization}
This project is carried out in collaboration with the company \textit{Software Improvement Group (SIG)}, and it is motivated by the \textit{FASTEN} project \footnote{\url{https://www.fasten-project.eu/}}. The objective of this project is to improve the quality of open-source development environments to make them more secure and reliable. For this reason, the FASTEN project aims to analyze the software library dependencies that the projects have, in more detail. SIG is collaborating in the FASTEN project since the goals of the project, are aligned with those of SIG. The good maintenance of the dependencies of a project is also part of the heath and security of software applications.

\subsection{Research questions} % Here, I still have to think about the terminology used, because I am not sure about this "software product" thing.
To tackle the problems described in the previous section, we specify the following research questions:

\blankl
\textbf{RQ1:} \textit{How can we measure the degree of source code dependency between two software products?}

In order to define an approach to answer this question and to divide the question in different parts, we have defined the following subquestions:

\begin{itemize}
  \item \textbf{RQ1.1:} \textit{What constitutes a dependency between two products?}

  One of the first steps to find a way to measure the degree of dependency, it is to determine what creates a dependency. The type of connection between the products and how to count them, among other characteristics.
  \blankl

  \item \textbf{RQ1.2:} \textit{Which metrics can be used to measure the dependency?}

  With this question, we investigate which existing coupling metrics could be used to measure dependency between software products. This implies evaluate the metrics according to the characteristics defined in the previous subquestion, to select the ones applicable for our case. In addition, it is also necessary to define how to adapt them to the new usage.
  \blankl

  \item \textbf{RQ1.3:} \textit{How can we calculate the metrics?}

  With a defined set of metrics to use, it is necessary to calculate them, to have results to evaluate the metrics. This is done with a call-level dependency graph, as will be explained later on.
  \blankl

  \item \textbf{RQ1.4:} \textit{Which is the meaning of the metrics in this scenario?}

  After calculating the metrics in a project, there is a number as a result. But what is the meaning of the number? What is the meaning of the metric? The goal is to define the meaning of the metric, so a number gives actionable information.
  \blankl

  \item \textbf{RQ1.5:} \textit{How can we validate the proposed metrics?}

  Once with the results of the Finally, with the results, and the definition of the metrics, we have validated them. There are two approaches to do so, the theoretical and the empirical. The metrics will be validated using both approaches, as extensibly as time permits.
\end{itemize}

\blankl
\textbf{RQ2:} \textit{How can we measure the effort required to replace the use of a library in a software product?}

This second question is in the case that it is necessary to replace the dependency of the product with a different library. To do so, it could be useful to have an estimation of the effort needed to replace the library. To answer this question, we have defined the following subquestions:

\begin{itemize}
  \item \textbf{RQ2.1:} \textit{What do we consider effort, which is the definition?}

  There are many different definitions of effort in the literature. One of the most common ones is the man-hours. In this case we are going to define which one is the effort measured in this research.
  \blankl

  \item \textbf{RQ2.2:} \textit{Which is the definition of replace a dependency?}

  There are many different cases which could be defined as replacing a dependency. Therefore, we are going to specify the taxonomy used to describe the types of library replacement, and from those types, determine which are the studied ones, and which are not.
  \blankl

  \item \textbf{RQ2.3:} \textit{Which are the applicable methods to measure effort?}
  From all the methods and strategies to measure effort existing in the literature, we are going to chose which are the ones that measure the effort that we defined in RQ2.1, and are applicable to the specific situation defined in RQ2.2.
  \blankl

  \item \textbf{RQ2.4:} \textit{How can we validate the measurement of the effort?}
  Finally, as in RQ1.5, it is necessary to validate the method used to measure effort, both empirically and theorically. Case studies are used for the empirical validation, comparing the real effort with the estimated one.
\end{itemize}

\blankl
\textbf{RQ3:} \textit{How can the degree of the dependency and the effort needed to replace a software product be aggregated?}

This last Research Question is meant to aggregate the research done in RQ1 and RQ2.

\subsection{Research method}
The main research method that is going to be employed during this project is \textit{Technical Action Research} (TAR) \cite{wieringa2012technical}.
This research method is artifact-based, which means that the first step is to produce the artifact meant to be used in certain situations envisioned by the researcher. The testing of this artifact, to see if it is effective in these situations, is done through a number of iterations. First, under ideal conditions, and then changing the experiments step by step to reach a real-world situation. These iterations, in the context of this project, are going to be limited due to the existing deadlines. However, there is the option of continuing with this part of the work in the future.

\blankl
In addition, the research will also include controlled experiments. These experiments will be conducted as the empirical part of the validation of the coupling metrics. Additionally, the empirical validation of the effort estimation is going to be achieved by the means of case studies. In these case studies, the estimated effort will be compared with the real effort that the change required.

\blankl
Some of the most difficult parts of this research are the validation and adaptation of the coupling metrics.
First, the validation since there is not a unique way to do it, but rather a wide variety of criteria and approaches. Therefore, conducting a complete validation is a complex task as it is not possible to validate every aspect of the metric.
In addition, even though the metrics that are going to be used to measure coupling between software products already exist, they are going to be adapted. In order to use the metrics for coupling between packages, it is necessary to define the aggregation method of the metrics. These changes of the metrics could be complex for some of them. However, there are many metrics that could be used, and the ones that cannot be adapted could serve as an indication of the types of dependencies that could exist when adapting other metrics.

\blankl
After the formal definition and theoretical validation of the metrics for both, the coupling measurement and the estimation effort, to replace a library, a Proof-of-Concept (PoC) will be made. With this PoC, the metrics will be calculated for the projects given to the PoC. Once the PoC is ready, the empirical validation will be carried out for each of the metrics, by conducting controlled experiments and case studies as elaborated above.
The methodology for validating the metrics chosen during this research will be divided into two phases: a first phase of theoretical validation and a second one of empirical validation \cite{srinivasan2014software}.

\section{Contributions}
Considering the current state of the art in the domain of this thesis project, the main contributions made by this research are the following:

\begin{enumerate}
	\item Applying coupling metrics to measure the coupling between two different software products, redefining the meaning and usage. Therefore, creating a methodology to measure the dependencies in a project.
	\item Proposing a model to calculate the effort needed to replace a library usage from a software product.
	\item Evaluation and validation of the proposed metrics and model, by the means of a proof-of-concept tool to calculate the metrics.
\end{enumerate}

\section{Scope} % I think this is the problem analysis, but I'm not so sure


\section{Outline}
In Chapter \ref{ch:Background} we describe the background of this thesis based on the existing literature of the domain.
Chapter \ref{ch:TheoricModel} describes the theorical model of the metrics used to calculate the degree of dependency between software products and the effort needed to replace a library in a project.
The theorical model is used in Chapter \ref{ch:PoC}, which describes the creation of the Proof-of-Concept of this model.
In Chapter \ref{ch:Experiments}, the set up and execution of the experiments is explained, and the results of the experiments are shown. These results are discussed in Chapter \ref{ch:Discussion}. Chapter \ref{ch:RelatedWork}, contains the work related to the domain of this thesis.
Finally, we present our concluding remarks in Chapter~\ref{ch:Conclusion} as well as the future work.
