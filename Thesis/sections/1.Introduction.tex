% !TEX root = ../main.tex
\chapter{Introduction}\label{ch:Introduction}

\section{Problem statement}

\subsection{Context}
Currently, there are many open-source libraries available for developers to reuse the features that these libraries implement. This practice is becoming more and more popular since it allows to reuse previously developed code, and helps developers avoid implementing the same functionalities multiple times.

When a developer uses an open-source library in a project, it creates a dependency between the project and the library. This implies that a significant number of projects depend on other open-source projects, and it adds the task of managing these dependencies to the maintenance tasks of the project.  Proper maintenance of the dependencies of a project is also part of the health and security of software applications, and it is one of the problems that the field of software engineering is trying to solve \cite{kula2014visualizing}. For instance, an update of a dependency can involve changing part of the code, if the update contains breaking changes \cite{Raemaekers2017}.

The management and maintenance of the dependencies of a project is an important task. External libraries, just like any other software product, can have security vulnerabilities that may affect the projects that depend on these libraries. For example, some vulnerabilities can become security problems that can have a negative impact in terms of integrity, privacy or availability.

\blankl
Currently, developers have package managers at their disposal, to ease the task of managing the dependencies of their projects. However, the dependency management available in these package managers only evaluate if a dependency exists or not and a more detailed evaluation is missing \cite{hejderup2018prazi}. For instance, here is no way to evaluate how much a project depends on a library.

\blankl
Therefore, this thesis has the aim to create a model to evaluate the dependencies, in order to obtain information of the \textit{actual usage} of the dependencies. A set of metrics is proposed to measure the dependencies between projects and the dependencies these have. The metrics are designed to evaluate the dependencies according to two different perspectives, the code affected by the dependency, and how much of a dependency is used.

\subsection{Host organization}
This project has been carried out in collaboration with the company \textit{Software Improvement Group (SIG)}, and it is motivated by the \textit{FASTEN} project \footnote{\url{https://www.fasten-project.eu/}}. The objective of this project is to improve the quality of open-source development environments to make them more secure and reliable. For this reason, the FASTEN project aims to analyze the software library dependencies that the projects have in more detail.

\subsection{Research questions}
\discussion{Still deciding whether to use software product, or libraries, how to introduce it. Because, for the PoC I am using only libraries available in Maven, but the model is meant to be used to measure the dependencies with any type of software product}
To tackle the problems described in the previous section, we specify the following research questions:

\blankl
\textbf{RQ1:} \textit{How can we measure the degree of code dependency between two software products with a direct dependency?}

\blankls
With this question, we want to propose a set of metrics to measure a dependency from the point of view of the product that has a dependency with another product. How much is it affected by the dependency.

\begin{itemize}
  \item \textbf{RQ1.1:} \textit{What constitutes a dependency between two products?}

  First, we need to determine what creates a dependency. The type of connection between the products and how can it be measured.

  \item \textbf{RQ1.2:} \textit{Which metrics can be used to measure the dependency?}

  We propose metrics to measure the dependency described. Existing metrics are considered, as well as new ones.

  \item \textbf{RQ1.3:} \textit{How can the proposed metrics be validated?}

  There are many different approaches to validate metrics, some of which are used to carry out the validation of the proposed metrics.
\end{itemize}

\blankl
\textbf{RQ2:} \textit{How can we measure the degree of code dependency between two software products with a transitive dependency?} \unsure{I'm not sure if I should phrase this question like this, or focusing on the differences: How should the measurements of the dependencies be adapted for the case of transitive dependencies}

\blankls
Transitive dependencies involve more factors than the direct dependencies. Therefore, the metrics proposed for the direct dependencies have to be adapted for the transitive ones.

\blankl
\textbf{RQ3:} \textit{How can we mesure how much is a dependency used by a software product?}

\blankls
For this question, we look at the dependency from a different perspective. In this case, we measure how much of the dependency is being used by the software product.

\blankl
\textbf{RQ4:} \textit{How can we visualize the metrics designed to model the software dependencies?}

\blankls
We present a few visualization to see the metrics of the model. The visualizations are presented to software developers to discuss the usefulness and actionability.

\subsection{Research method}
The main research method that is going to be employed during this project is \textit{Technical Action Research} (TAR) \cite{wieringa2012technical}.
This research method is artifact-based, which means that the first step is to produce the artifact meant to be used in certain situations envisioned by the researcher. The testing of this artifact, to see if it is effective in these situations, is done through a number of iterations. First, under ideal conditions, and then changing the experiments step by step to reach a real-world situation. These iterations, in the context of this project, are going to be limited due to the existing deadlines. However, there is the option of continuing with this part of the work in the future.

\blankl
In addition, the research will also include experiments. These experiments will be conducted as the empirical part of the validation of the metrics and of the proof-of-concept implementation. Additionally, the empirical validation of the effort estimation is going to be achieved by the means of case studies. In these case studies, the estimated effort will be compared with the real effort that the change required. \discussion{Lodewijk: this seems a bit ambitious; you may replace this with the application of the measurements to one or more concrete cases (systems)}

\blankl
Some of the most complex parts of this research are the formulation and validation of the metrics to measure the dependencies.
First, the validation since there is not a unique way to do it, but rather a wide variety of criteria and approaches. Therefore, conducting a complete validation is a complex task as it is not possible to validate every aspect of the metric.
In addition, even though some the metrics that are going to be considered to measure coupling between software products already exist, they are going to be adapted.

\blankl
After the formal definition and theoretical validation of the metrics, a Proof-of-Concept (PoC) will be made. With this PoC, the metrics will be calculated for the projects given to the PoC. Once the PoC is ready, the empirical validation will be carried out for each of the metrics, by conducting experiments and case studies as elaborated above.

\section{Contributions} % TODO: revisit when all the practical stuff is done
Considering the current state of the art in the domain of this thesis project, the main contributions made by this research are the following:

\begin{enumerate}
	\item Applying coupling metrics to measure the coupling between two different software products, redefining the meaning and usage of the metrics. Creating a model to measure the direct and transitive dependencies of a software product.
  \blankls

	\item Evaluation and validation of the proposed metrics and model, by the means of a proof-of-concept tool to calculate the metrics.
\end{enumerate}

\section{Scope}\label{section:scope}
\subsection{Quantifying the dependencies}
To the best of our knowledge, there are no papers about measuring the degree of dependency between two separate projects. However, it is true that the degree of dependency between two classes or modules of the same project has already been measured many times, using coupling metrics \cite{briand1999unified}.

Therefore, we propose re-using the already existing coupling metrics, meant to measure the coupling between units of the \textit{same} project and adapt them to measure coupling \textit{between} projects.

\blankl
According to Poshyvanyk and Marcus in \cite{poshyvanyk2006conceptual}, there are six main groups of coupling metrics:

\begin{itemize}
  \item \textbf{Structural coupling metrics:} Measured directly from static source code analysis. Largely studied by the literature about coupling.

  \item \textbf{Dynamic coupling measures:} Measured using dynamic code analysis. \textit{"Introduced as the refinement to existing coupling measures due to gaps in addressing polymorphism, dynamic binding, and the presence of unused code by static structural coupling measures"} \cite{poshyvanyk2006conceptual}.

  \item \textbf{Evolutionary and Logical coupling:} According to Zimmermann and Diehl \cite{zimmermann2005mining}, evolutionary coupling can: \textit{"tell us which parts of the system are coupled by common changes or cochanges."}

  \item \textbf{Coupling measures based on information entropy approach:} Coupling metrics based on the information-theory approach, such as the metrics proposed by Allen and Khoshgoftaar in \cite{allen1999measuring}.

  \item \textbf{Conceptual coupling metrics:} Based on the semantic similarity between the elements. This is the focus of the work from Poshyvanyk and Marcus \cite{poshyvanyk2006conceptual}.

  \item \textbf{Coupling metrics for specific types of software applications:} Specialized coupling metrics for certain types of projects, such as knowledge-based systems or aspect-oriented approach.
\end{itemize}

\blankl
Since this research is aimed to be independent of the domain, the last cathegory is not considered in this thesis. Moreover, the evolutionary coupling is not possible to be applied in our context since it is very likely that the separate projects are not going to evolve at the same time, given that they are not developed by the same team. Finally, the research of this project, owing to the time limitation, is going to be centered on the structural metrics.

These metrics are going to be proposed as a first step to measure the degree of library dependency. Nevertheless, the metrics can be extended and calculated more accurately by adding dynamic coupling and information entropy approach metrics if time permits or in future work.


\subsection{Estimating the effort to replace a dependency}
% Write after deciding what to do about it

\begin{comment}
  In order to estimate the effort needed to replace a dependency in a project, there are different variables that need to be considered first:

  - How much code do I need to change - estimate based on the usage of the Dependency
  - Refactoring adjustment
  - How many methods do I need to change.
  - The impact of the change - inside the project.
\end{comment}

\subsection{Evaluation and validation} \unsure{what is the purpose of this section??? This feels a bit like 'the approach}
This thesis includes the evaluation and validation of the proposed methodology to measure dependencies and estimate the effort to replace dependencies. To do so, it is necessary to use in practice the proposed measurements. Therefore, this research includes a Proof of Concept (PoC) to use the proposed model with real projects. Then, the metrics and estimations are validated using the results obtained with the PoC.
To calculate the metrics with the PoC, it is necessary to create a dependency graph of the project that is going to be evaluated. This graph has to be a call-level dependency graph since we require a fine-grained analysis of the source code dependencies. Moreover, it has to represent the software ecosystem of the project considering the various versions of the libraries it depends on.

Ultimately, the metrics used in this project will have to be validated. There is not a unique way to validate metrics which is globally accepted and used. Therefore, various approaches will be adopted to validate the metrics as holistically as possible. In the paper \cite{srinivasan2014software}, Srinivasan et al. explain that there are two fundamental approaches for metric validation: \textit{theoretically} and \textit{empirically}. Therefore, to provide a check of validity that is as broad as possible, a mixture of these two approaches will be used during this project.

For the theoretical validity of the coupling metrics, these are validated according to the \textit{Mathematical Properties of Measures for Coupling} \cite{srinivasan2014software}.
The empirical validation of the metrics is carried out by the means of case studies. For instance, the predicted effort corresponds to the real required effort, within a certain error window. \unsure{this seems right, but different from 1.1.4?}

\section{Outline}
In Chapter \ref{ch:Background} we describe the background of this thesis based on the existing literature of the domain.
Chapter \ref{ch:TheoreticModel} describes the theoretic model of the metrics used to calculate the degree of dependency between software products and the effort needed to replace a library in a project.
The theoretic model is used in Chapter \ref{ch:PoC}, which describes the creation of the Proof-of-Concept of this model.
In Chapter \ref{ch:Experiments}, the set up and execution of the experiments is explained, and the results of the experiments are shown. These results are discussed in Chapter \ref{ch:Discussion}. Chapter \ref{ch:RelatedWork}, contains the work related to the domain of this thesis.
Finally, we present our concluding remarks in Chapter \ref{ch:Conclusion} as well as the future work.
