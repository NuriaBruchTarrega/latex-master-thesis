% !TEX root = ../main.tex
\chapter{Introduction}\label{ch:Introduction}

\section{Problem statement}
Currently, there are many libraries available for developers to reuse the features that these libraries implement, both commercial and open-source. This practice is becoming more and more popular since it allows the reuse of previously developed code and helps developers avoid implementing the same functionalities multiple times. For example, the Maven Repository Central contained 2.8M artifacts in 2018 \cite{Benelallam2019}.

When a developer uses a library in a project, it creates a dependency between the project and the library. It implies that a significant number of projects depend on other libraries, the Maven Repository Central includes more than 9M dependencies between artifacts, dated 2018 \cite{Benelallam2019}. This adds the task of managing these dependencies to the maintenance tasks of the project. Proper maintenance of the dependencies of a project is also part of the health and security of software applications, and it is one of the problems that the field of software engineering is trying to solve \cite{kula2014visualizing}. For instance, an update of a dependency can involve changing part of the code, if the update contains breaking changes \cite{Raemaekers2017}.

The management and maintenance of the dependencies of a project is an important task. External libraries, just like any other software product, can have security vulnerabilities that may affect the projects that depend on these libraries. For example, some security vulnerabilities can become problems that can have a negative impact in terms of integrity, privacy, or availability.

\blankl
Currently, developers have package managers at their disposal, to ease the task of managing the dependencies of their projects. However, the dependency management available in these package managers only evaluates if a dependency exists or not and a more detailed evaluation is missing \cite{hejderup2018prazi}. For instance, there is no way to evaluate how much a project depends on a library.

\blankl
Therefore, this thesis aims to create a model to evaluate the dependencies, in order to obtain information of the \textit{actual usage} of the dependencies. A set of metrics is proposed to measure the dependencies between projects and the dependencies these have. The metrics are designed to evaluate the dependencies according to two different perspectives, the code affected by the dependency, and how much of a dependency is used.

This project has been carried out in collaboration with the company \textit{Software Improvement Group (SIG)}, and it is motivated by the \textit{FASTEN} project \footnote{\url{https://www.fasten-project.eu/}}. The objective of this project is to improve the quality of open-source development environments to make them more secure and reliable. For this reason, the FASTEN project aims to analyze the software library dependencies that the projects have in more detail.

\subsection{Research questions}
To tackle the problems described in the previous section, we specify the following research questions:

\blankl
\textbf{RQ1:} \textit{How can we measure the degree of code dependency between two software products with a direct dependency?}

\blankls
With this question, we want to propose a set of metrics to measure a dependency from the point of view of the product that has a dependency with another product. How much it is affected by the dependency.

\begin{itemize}
  \item \textbf{RQ1.1:} \textit{What constitutes a dependency between two products?}

  First, we need to determine what creates a dependency. The type of connection between the products and how can it be measured.

  \item \textbf{RQ1.2:} \textit{Which metrics can be used to measure the dependency?}

  We propose metrics to measure the dependency described. Existing metrics are considered, as well as new ones.

  \item \textbf{RQ1.3:} \textit{How can the proposed metrics be validated?}

  There are many different approaches to validate metrics, some of which are used to carry out the validation of the proposed metrics.
\end{itemize}

\blankl
\textbf{RQ2:} \textit{How can we measure the degree of code dependency between two software products with a transitive dependency?}

\blankls
Transitive dependencies involve more factors than direct dependencies. Therefore, the metrics proposed for the direct dependencies have to be adapted for the transitive ones.

\blankl
\textbf{RQ3:} \textit{How can we measure how much is a dependency used by a software product?}

\blankls
For this question, we look at the dependency from a different perspective. In this case, we measure how much of the dependency is being used by the software product.

\blankl
\textbf{RQ4:} \textit{How can we visualize the metrics designed to model the software dependencies?}

\blankls
We present a few visualizations to see the metrics of the model. The visualizations are presented to software developers to discuss the usefulness and actionability.

\subsection{Research method}
The main research method that is going to be employed during this project is \textit{Technical Action Research} (TAR) \cite{wieringa2012technical}.
This research method is artifact-based, which means that the first step is to produce the artifact meant to be used in certain situations envisioned by the researcher. The testing of this artifact, to see if it is effective in these situations, is done through a number of iterations. First, under ideal conditions, and then changing the experiments step by step to reach a real-world situation. In the context of this master thesis, because of time constraints, performing multiple iterations is not possible. However, there is the option of continuing with this part of the work in the future.

\blankl
In addition, the research will also include experiments. These experiments will be conducted as the empirical part of the validation of the metrics and the proof-of-concept implementation, of both the model and the proposed visualizations.

\section{Contributions}
Considering the current state of the art in the domain of this thesis project, the main contributions made by this research are the following:

\begin{enumerate}
	\item Defining a model for software dependencies including both, direct and transitive dependencies of a software product. The model includes modified coupling metrics to measure the coupling between two different software products, redefining the meaning and usage of the metrics.
  \blankls

	\item Evaluation and validation of the proposed metrics and model, by the means of a proof-of-concept tool to calculate the metrics.

  \item A proposed visualization of the model, validated through expert review.
\end{enumerate}

\section{Outline}
In Chapter \ref{ch:Background} we describe the background of this thesis based on the existing literature of the domain.
Chapter \ref{ch:TheoreticModel} describes the metrics used to model the dependencies between software products.
The theoretic model is used in Chapter \ref{ch:PoC}, which describes the creation of the proof-of-concept of this model.
In Chapter \ref{ch:Experiments}, the set up and execution of the experiments is explained, and the results of the experiments are shown. These results are discussed in Chapter \ref{ch:Discussion}. Chapter \ref{ch:RelatedWork}, contains the work related to the domain of this thesis.
Finally, we present our concluding remarks in Chapter \ref{ch:Conclusion}, as well as future work.
