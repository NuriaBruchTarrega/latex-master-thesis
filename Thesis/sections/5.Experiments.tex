% !TEX root = ..\main.tex
\chapter{Experiments}\label{ch:Experiments}

\section{Experiment 1: Comparison}\label{sec:Exp1}
The goal of this experiment is to provide validation of the implementation in the PoC. This is done with the results of the paper \textit{"A Comprehensive Study of Bloated Dependencies in the Maven Ecosystem"} by Soto-Valero et al. \cite{soto2020comprehensive}. In this work, a byte-code analysis is performed on Maven Artifacts to detect which of the dependencies of this artifacts are \textit{bloated}, which we refer to as \textit{unused}. Although the PoC created in this thesis does not perform exactly the same kind of analysis, we consider that a dependency is \textit{unused} if no usage is detected by any of the metrics in the model.

\subsection{Experimental set up}
Soto-Valero et al. perform a qualitative analysis, in which they analyze 31 libraries available as Maven artifacts. If unused dependencies are found during the analysis, a \textit{Pull Request} is made to the \textit{GitHub} repository of the artifact in which the unused dependencies are deleted from the \textit{pom} file.

With the information in the paper, we collect the \textit{GroupId} and \textit{ArtifactId} of the 31 artifacts. Out of the 31, 2 could not be found in the \textit{Maven Repository Central}. For the other 29, the \textit{version} to use in the experiment is determined by finding the last version released before the experiment by Soto-Valero et al. was conducted - November of 2019.

Of the 29 artifacts, 13 could not be used with the PoC, because either the artifact itself or some dependency could not be downloaded from \textit{Maven Central}, leaving a set of 16 libraries to analyse and compare the results with the results obtained by Soto-Valero et al. The table containing the identifiers of the artifacts used in this experiment can be found in Table \ref{table:comparison-artifacts}.

\begin{table}[ht]
    \begin{center}
    \begin{tabular}{|l|l|l|l|l|l|l|l|l|}
    \hline
    Group Id              & Artifact Id                     & Version       \\
    \hline
    org.mybatis           &	mybatis	                        & 3.5.3         \\
    org.apache.flink      & flink-core                      & 1.9.1         \\
    com.puppycrawl.tools  & checkstyle                      & 8.27          \\
    com.google.auto       & auto-common                     & 0.10          \\
    edu.stanford.nlp      & stanford-corenlp                & 3.9.2         \\
    com.squareup.moshi    & moshi-kotlin                    & 1.9.2         \\
    org.neo4j             & neo4j-collections               & 3.5.13        \\
    org.asynchttpclient   & async-http-client               & 2.10.4        \\
    org.alluxio           & alluxio-core-transport          & 2.1.0         \\
    com.github.javaparser & javaparser-symbol-solver-logic  & 3.15.5        \\
    io.undertow           & undertow-benchmarks             & 2.0.27.Final  \\
    org.teavm             & teavm-core                      & 0.6.1         \\
    com.github.jknack     & handlebars-markdown             & 4.1.2         \\
    ma.glasnost.orika     & orika-eclipse-tools             & 1.5.4         \\
    fr.inria.gforge.spoon & spoon-core                      & 8.0.0         \\
    org.jacop             & jacop                           & 4.7.0         \\
    \hline
    \end{tabular}
    \end{center}
    \caption{Identifiers of the Maven artifacts used for comparison}
    \label{table:comparison-artifacts}
\end{table}

The first idea was to create a request that computed the comparison automatically. However, finding the reason why the results were not exactly the same, in most of the cases needed manual checking and research. That is why, the analysis of the libraries in Table \ref{table:comparison-artifacts} have been executed one by one, and the comparison has been done manually.

\subsection{Results}

- how many libraries have different results and why.

- overall evaluation

\section{Experiment 2: Relevance of the coupling metrics}

The goal of this experiment is to validate if the coupling metrics designed in the model, namely \texttt{MIC}, \texttt{AC}, \texttt{TMIC}, and \texttt{TAC}, are a good indicator of the usage of the dependencies by the clients. As a partial validation of the relevance of the metrics, we compare it with the results gathered from the usage metrics. We want to know how often it happens, that a dependency is used, either by using classes or methods, and it is detected as uncoupled by the coupling metrics. This way, we know if there are many cases in which a dependency that is only used with a type of connection other than method invocation or field declaration.

\unsure{I'm not sure if this is the right place for the explanation of the failed data recolection, or if it is too long}

The original idea was to measure real-world data about how the clients update the dependencies and their impact on the code. We could then have seen the correlation of this impact with the degree of dependency measured with the coupling metrics. Different approaches were taken to obtain real-world data.

First, we tried to find in GitHub commits in which there had been an update of a dependency. However, the search engine in GitHub does not allow to filter the results by the language of the commit. Therefore, most of the results obtained were not useful. Also, most of the updates are only patches, which require only a bump in the version number of the declared dependency.

Based on these findings, the second approach we took was to look for updates that contained breaking changes. To find the libraries that had these type of changes, and in which versions, we used the \textit{Maven Dependency Dataset} \cite{Raemaekers2013}. Raemaekers et al. used this dataset to analyze the use of semantic versioning and the possible impact of breaking changes \cite{Raemaekers2017}. It is possible to query this dataset to obtain libraries with breaking changes, with version numbers and other libraries that depended on these. However, we need to find the commit of the client library in which the update containing a breaking change was made, and it is not always possible. We considered some of the requirements to be able to analyze a dependency with the PoC. For instance, we need all the dependencies of the client library available in Maven, and testing dependencies cannot be used since they are not analyzed by the tool. Considering all these requirements, it was not possible to obtain enough data for the experiment from the \textit{Maven Dependency Dataset} on time, since all these checks had to be done manually.

Next, we contacted the first author of the paper \textit{"Why and How Java Developers Break APIs"} \cite{Brito2018}, which mines GitHub repositories to find possible breaking changes in APIs, to obtain the dataset of breaking changes created based on their findings. Brito, the author, shared the dataset with us. The dataset includes 24 commits containing breaking changes, which correspond to 19 different libraries. Out of the 25 commits, 12 are from libraries managed by Gradle instead of Maven and cannot be used with the PoC. Besides, we could not find 4 of the commits in GitHub, and 2 others correspond to testing libraries, which are out of the scope of the analysis performed by the PoC. Therefore, there were only 6 breaking changes left, for which 3 the Maven artifact that these belong to had no dependants for which to do the analysis. The last 3 have only one dependant, and therefore is not possible to compare the impact of the breaking changes.

Finally, we tried to manually search for deprecated libraries and other libraries that used them — however, similar problems where encountered. Finding commits which replaced a deprecated dependency and the client library and all the dependencies are available in \textit{Maven Central Repository}, is a manual task that, after multiple hours of work, gave no results.


\subsection{Experimental set up}
To run this experiment, we prepared a new request in the API of the PoC. The request has to contain a path to a txt file (tab delimited). The file has to contain three columns (with headers): Group Id, Artifact Id, and version. For each one of the rows, the analysis of the dependencies is performed. The result of each of the analysis is processed, summarizing all the analysis with the following information:

\begin{itemize}
  \item \textbf{Total number of dependencies:} Number of dependencies of all the analyzed client libraries, including both, direct and indirect.
  \item \textbf{Times coupling metrics were not enough:} Number of dependencies for which all the coupling metrics had value zero, but there where methods and classes found reachable by the usage metrics.
  \item \textbf{Times MIC/TMIC were not enough:} Number of times in which there was usage found, but MIC (or TMIC in the case of transitive dependencies) had value zero.
  \item \textbf{Times AC/TAC were not enough:} Number of times in which there was usage found, but MIC (or TMIC in the case of transitive dependencies) had value zero.
  \item \textbf{List server libraries coupling metrics not enough:} The list of \textit{GroupId}, \textit{ArtifactId}, and \textit{version} of the server libraries for which all the coupling metrics were not enough to indicate if there is usage or not.
  \item \textbf{List server libraries MIC/TMIC not enough:} The list of \textit{GroupId}, \textit{ArtifactId}, and \textit{version} of the server libraries for which the metrics \texttt{MIC} and \texttt{TMIC} were not enough to indicate if there is usage or not.
  \item \textbf{List server libraries AC/TAC not enough:} The list of \textit{GroupId}, \textit{ArtifactId}, and \textit{version} of the server libraries for which the metrics \texttt{AC} and \texttt{TAC} were not enough to indicate if there is usage or not.
\end{itemize}

As can be seen, in addition to the number of times that a dependency was used and it was not detected by the metrics (or at least by one of them), the list of server libraries of these dependencies is also stored. This way, it is possible to analyze which types of libraries are those, and why the coupling metrics are not relevant in this case.

\blankl
The experiment was run with a file containing 49 client libraries from the \textit{Maven Central Repository}.  We selected the client libraries to use for this experiment with the following criteria. First, we used the same libraries as in the comparison experiment (see Section \ref{sec:Exp1}), but using the last version of each library. We decided to reuse these libraries because the criteria used to select these libraries by Soto-Valero et al. \cite{soto2020comprehensive} is aligned with the needs of this experiment, and are listed below:

\begin{itemize}
  \item The library is relevant - has more than 100 stars on GitHub.
  \item The library can be built successfully with Maven.
  \item Has been developed recently - in the case of Soto-Valero et al. at least October 2019.
  \item The library has at least one dependency declared.
  \item It is indicated how to create a pull request.
\end{itemize}

Although some of the items of the list are not explicitly required for our experiment. We need that the library can be built and is available in Maven as well as that it has at least one relevant dependency (compile scope). Therefore, the libraries in this set are a good fit for the experiment.

To analyze more client libraries and, therefore, more dependencies, we extended the list of libraries. First, we visited the popular libraries list of the \textit{Maven Central Repository} \footnote{\url{https://mvnrepository.com/popular}}. In addition, we queried the dataset generated by Harrand et al. \cite{Harrand2019} with the 99 most popular libraries from Maven, according to the number of clients these have. For each of the libraries, we selected the last version available in Maven, and filtered the resulting list according to the following criteria:

\begin{itemize}
  \item The artifact of the last version of the library should have at least one dependency with scope compile.
  \item The artifact and all its dependencies can be obtained from the \textit{Maven Central Repository}.
\end{itemize}

\unsure{Should I show the exact content of the file? In case I should, should I do it in a Table here or in an appendix at the end of the document?}

\subsection{Results}

\section{Experiment 3: Sensitivity Analysis}
As explained before, we have not been able to obtain the real-world data to understand how the impact of the transitive dependencies behaves, and correlate it with our transitive coupling metrics \texttt{TMIC}, and \texttt{TAC}. Therefore, we cannot know which would be the actual value of the \textit{propagation factor} for these two metrics. Instead, what we do is a sensitivity analysis of the \textit{propagation factor} on these two metrics.

A \textit{Sensitivity Analysis} consists on analyzing how much an input variable affects the output. In this case, the input variable is the \textit{propagation factor} and the output is the value of the metrics.

\subsection{Experimental set up}

To run this analysis, we have set up a new request in the API of the PoC. This request receives a list of Maven artifacts, in a \textit{.txt} file (tab delimited). The file includes three columns, containing for each artifact, the following information: \textit{group id}, \textit{artifact id}, and \textit{version}.

The first step is to run the calculation of the dependency model, for each of the dependencies of the artifacts. Then, for each one of the transitive dependencies, for which coupling was found, we run the sensitivity analysis. Since the \textit{propagation factor} is a value in the range $(0,1)$, we calculate the value of the metrics incrementing the propagation factor by $0.01$ from $0.01$ to $0.99$.

The content of the file which we have used to run the sensitivity analysis can be found in Table \cn{table:sensitivity-analysis-libraries}.

\subsection{Results}

\section{Experiment 4: Expert Interviews}
This last experiment has various goals. The main one is to validate the design of the visualization. According to Munzner \cite{Munzner2009}, there are four levels at which this validation can be done:

\begin{enumerate}
  \item Domain Problem and Data Characterization
  \item Operation and Data Type Abstraction
  \item Visual Encoding and Interaction Design
  \item Algorithm Design
\end{enumerate}

In this case, we focus on the third option: Visual Encoding and Interaction Design. To carry out this validation, we designed an \textit{Expert Review} by means of interviews. In addition, we included questions about the clarity and actionability of the metrics of the model. Therefore, these two aspects of the metrics, which are included in the set of validation criteria defined by Meneely et al. \cite{Meneely2012}, are also validated.

\subsection{Experimental set up}
The interview consists on 19 and a demonstration of the PoC, with two proposed scenarios in which the interviewee uses the tool. The questions are divided in four sections, dividing the interview in a total of five parts:

\begin{enumerate}
  \item \textbf{Demographics:} The questions of this section are related to the professional experience and current job of the interviewee.
  \item \textbf{Dependency Management:} In this part, the questions are focused on the experience of the interviewee with dependency management and the tools used for this purpose.
  \item \textbf{Demonstration:} The third part is the demonstration of the tool, in which two scenarios are presented to the interviewee. During the scenarios discussion, the interviewee has the control of the mouse, so the person can interact directly with the tool.
  \item \textbf{Visualizations:} The section after the demonstration contains questions about the tool itself and the designed visualizations.
  \item \textbf{Metrics:} The last section is focused on the designed metrics, and the clarity and comprehensibility of these, as well as the actionability.
\end{enumerate}

The interviews contain three types of questions: open answer, binary, and scaled from 1 to 5. During every question, even the binary and scaled questions, the interviewee has the option of making comments or discuss the answer. The list of questions contained in the interview, can be found in Table \ref{table:interview-questions}.

\begin{table}[p]
    \begin{center}
    \begin{tabularx}{\textwidth}{|X|l|l|}
    \hline
    Question & Section & Type \\\hline
    \hline
    1.  What is your software development role?  & Demographics & Open answer \\\hline
    2.	How many years of experience do you have as a software developer? & Demographics & Open answer \\\hline
    3.	Which programming language(s) do you usually use in your job? & Demographics & Open answer \\\hline
    4.	Which type of projects do you usually work on? & Demographics & Open answer \\\hline
    \hline
    5.	Do you have experience with dependency management? & Dependency Management & Binary \\\hline
    6.	To what extent is it important to you (or do you try) to have the dependencies up to date? & Dependency Management & Scaled \\\hline
    7.	To what extent is it important to you to monitor the vulnerabilities that your dependencies may be exploiting? & Dependency Management & Scaled \\\hline
    8.	Which tools (if any) do you use for dependency management? & Dependency Management & Open answer \\\hline
    9.	To what extent do you think the tools you used so far are helping you to maintain your dependencies? & Dependency Management & Scaled \\\hline
    \hline
    Scenario 1: You are a new maintainer of the library \textit{org.apache.flink:flink-core}. Since you have not worked in the development of this library, you want to see how the dependency tree looks like. What would you look for? & Demonstration & Scenario \\\hline
    Scenario 2: You realize that a library called \textit{kryo} has a new version, which has been announced to contain breaking changes. How likely it would affect your library, and which classes are affected. & Demonstration & Scenario \\\hline
    \hline
    10.	How much do you agree that the tool is useful in the presented scenarios? & Visualizations & Scaled \\\hline
    11.	How much do you agree that managing dependencies would be easier with the presented tool? & Visualizations & Scaled \\\hline
    12.	With your job in mind, which (if any) are the most useful of the visualizations? & Visualizations & Open answer \\\hline
    13.	How much do you agree that the presented tool would be useful in your job? & Visualizations & Scaled \\\hline
    14.	Is there some other visualization or change you would like to see? For which cases do you think it would be useful? & Visualizations & Scaled \\\hline
    \hline
    15.	To what extent do you agree that the metrics are clear and comprehensible? (Answer per metric) & Metrics & Scaled \\\hline
    16.	To what extent do you agree that the metrics are useful in the described scenarios & Metrics & Scaled \\\hline
    17.	To what extent do agree that the metrics are actionable in the sense that they give you the information you need to make a decision? (Answer per metric) & Metrics & Scaled \\\hline
    18.	Which (if any) do you think are the most useful of the metrics? Based on the tasks that you usually do in your job. & Metrics & Open answer \\\hline
    19.	Is there some other metric or change that you would like to be added to the model? In which scenarios do you think it could be useful? & Metrics & Open answer \\\hline
    \end{tabularx}
    \end{center}
    \caption{Questions of the interview}
    \label{table:interview-questions}
\end{table}

The interviews are done via \textit{Zoom}\footnote{\url{https://zoom.us/}}. \textit{Zoom} offers the possibility of sharing the control of the mouse with other participants, as well as the option of recording the interview. The interviews are recorded to be able to rewatch it afterwards and take notes of the answers of the interviewees. Therefore, the interview itself feels more like a normal conversation, and there are no pauses. The control of the mouse was shared with the interviewees so they could try to use the tool themselves, to get a better idea of how it works and how they would use it.

\subsection{Results}
In this section, we show the answers obtained during the interviews. The results will be discussed in section \ref{sec:discussion-interviews}: the suitability of the visualizations, as well as the clarity and actionability of the metrics.

\subsubsection{Demographics}

The roles of the 15 participants in the interviews include: Software developer, Software engineer, Security consultant, Technology lead, Head of innovation, Head of development, and Head of product. The years of experience range from 1 to 20, with an average of 7.13. A half of the interviewees have worked in backend development, and web services systems. In addition, some of the other type of projects include mobile applications, and frontend development. The languages in which the interviewees have experience can be seen in Figure \ref{fig:interview-3}.

\begin{figure}[ht]
\begin{center}
\includegraphics[width=\textwidth]{figures/interview/Question3.png}
\caption{Answers to Question 3 of the interview}
\label{fig:interview-3}
\end{center}
\end{figure}

\subsubsection{Dependency Management}

The 15 interviewees have experience with dependency management. However, some of them indicated that it is not a task that they usually perform in their jobs, but rather in personal projects or from time to time. Figure \ref{fig:interview-6} shows the answers to question 6, regarding the importance of updating the dependencies. The reasons given by the interviewees answering \textit{Neutral} and \textit{Important} for not giving it more importance include: prioritizing the fact that the versions used are compatible, that there are no version incompatibilities with the current version used, and that the version used is stable.

\begin{figure}[ht]
\begin{center}
\includegraphics[width=\textwidth]{figures/interview/Question6.png}
\caption{Answers to Question 6 of the interview}
\label{fig:interview-6}
\end{center}
\end{figure}

 Figure \ref{fig:interview-7} the answers to question 7, about the importance of monitoring the vulnerabilities of the dependencies. The interviewees considering that the importance of monitoring the vulnerabilities of the dependencies was less than \textit{Very important} reasoned about it. For example, some said that it is something that they do, but not regularly. Also, they just update when there is a new version, to make sure that if a vulnerability has been discovered, the patch is always used. Finally, the last reason is that depends on the type of dependency that it is, since if it is not a customer-facing dependency, it is not that important.

\begin{figure}[ht]
\begin{center}
\includegraphics[width=\textwidth]{figures/interview/Question7.png}
\caption{Answers to Question 7 of the interview}
\label{fig:interview-7}
\end{center}
\end{figure}

In Figure \ref{fig:interview-8}, there are the answers to question 8. The interviewees gave more than one answer to the question, but always at least one explicitly included in the figure.

\begin{figure}[ht]
\begin{center}
\includegraphics[width=\textwidth]{figures/interview/Question8.png}
\caption{Answers to Question 8 of the interview}
\label{fig:interview-8}
\end{center}
\end{figure}

Finally, the answers to question 9, regarding the how helpul are the tools that the interviewees use for dependency management are. The interviewees that considered the tools to be really helpful (5), compared it to not using any tool at all. Whilts the interviewees giving lower marks (2-3), considered that there are features missing. Mainly, they considered that the basic needs are covered, however some more detailed information about how to manage the dependencies is not there.

\begin{figure}[ht]
\begin{center}
\includegraphics[width=\textwidth]{figures/interview/Question9.png}
\caption{Answers to Question 9 of the interview}
\label{fig:interview-9}
\end{center}
\end{figure}

\subsubsection{Visualizations}

\subsubsection{Metrics}

\section{Experiment 5: Benchmarking}
One of the main remarks received for the coupling metrics during the interviews, is that there is not a clear scale for those metrics. Therefore, the value of the metrics can be hard to interpret, since there is no indicator of which number is very high or very low.

Hence, we have done a benchmarking of the values of the metrics \texttt{MIC}, \texttt{AC}, \texttt{TMIC}, and \texttt{TAC}. The goal is to understand which is the distribution of the values, and be able to indicate which are the outliers of these metrics.

\subsection{Experimental set up}
To execute this experiment, we have set a new request in the backend of the PoC. This request should contain the path to a \textit{.txt} file which includes three different columns, tab delimited. Each row represents a Maven artifact, and for each column it indicates: \textit{group id}, \textit{artifact id}, and \textit{version}.

Then, the calculation of the metrics of the model is performed. For each analyzed dependency, the value of the benchmarked metrics is stored. The result consists on 4 different \textit{.csv} files. The first two, containing all the different values of the \texttt{MIC} and \texttt{AC} metrics respectively. The last files represent the values of the transitive coupling metrics, represented in three columns. The first, the dependency id, the second the distance, and the third the value calculated at the distance. Therefore, more than one row might be used to represent the \texttt{TMIC} or \texttt{TAC} of a dependency.

\subsection{Results}
