% !TEX root = ../main.tex
\chapter*{Abstract}
The usage of libraries, both commercial and open-source, provides the implementation of certain functionalities and is a widespread practice among developers. The usage of these libraries allows developers to avoid duplicating code by reusing it instead. However, when a developer uses a library in a software product, this creates a dependency. This dependency may result in transitive
dependencies when that library also depends on other libraries. The dependencies created when reusing a library can also carry problems. For example, if a library has a security issue, it can be propagated to the software product which depends on it, directly or indirectly. To deal with dependencies, developers can use package managers, which allow them to install and update the libraries they use. However, these package managers generally do a simple evaluation of the dependencies: either there is a dependency or not. Hence, a detailed evaluation of the dependencies is missing. More fine-grained information could be helpful for developers that deal with vulnerabilities, breaking changes, and deprecated dependencies.

In this thesis, we propose a model for software dependencies, which can help to provide a fine-grained evaluation. The model includes three types of metrics: coupling, coverage, and usage per class. For each metric in the model, we provide a formal definition and a theoretical validation by proving the metrics' properties. We additionally implemented a proof-of-concept tool which, given a library from the \textit{Maven Central Repository}, calculates the metrics of the model for each of the dependencies using bytecode analysis. Moveover, the proof-of-concept includes a visualization of the dependency tree, including the calculated metrics.

Finally, we conducted experiments to validate the model, the implementation of the proof-of-concept, and the visualization. The experiments include interviews with 15 professional developers who evaluated the clarity and actionability of the model’s metrics and the proposed visualizations.
