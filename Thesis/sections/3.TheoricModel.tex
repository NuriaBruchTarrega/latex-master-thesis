% !TEX root = ..\main.tex
\chapter{Theorical Model}\label{ch:TheoricModel}

\section{Dependency coupling}
% When defining what creates dependency, remember to talk about inheritance, polymorphy, interfaces and annotations

\section{Dependency replacement effort}
\subsection{Defining the taxonomy of package replacement (\textbf{RQ2.2})}

Before starting the measurement of package replacement, it is important to specify which kind of replacement is being studied. Therefore, we have defined a classification of the possible scenarios.

First, Kula et al. \cite{kula2014visualizing} in their study about the visualization of the evolution of library dependency, identify four distinct behabiours. Namely, adopter, idler, updater and dropper. In addition, Kikas et al. \cite{kikas2017structure} when studying the evolution of software ecosystems, define two different types of update, explicit and implicit. Considering this two classifications, we have created the following classification:

\begin{table}[ht!]
    \begin{center}
    \begin{tabularx}{\textwidth}{|X|l|l|}
      \hline
      Description & Kula et al. & Kikas et al. \\
      \hline\hline
      Add new package dependency & adopter & - \\
      \hline
      Maintain package dependency in new application version & idler & - \\
      \hline
      Manual update of dependent package version & updater & explicit package update \\
      \hline
      Automatic update of dependent package version & updater & implicit package update \\
      \hline
      Change a dependency from one package to another & dropper \& adopter & - \\
      \hline
      Remove dependency with a package & dropper & - \\
      \hline
    \end{tabularx}
    \end{center}
    \caption{Taxonomy of depdendency evolution}
    \label{table:taxonomy-dependency}
\end{table}

\smallskip\noindent
In this classification, we have added the scenario in which a dependency is dropped and substituted by another one. When looking at the terminology defined by Kula et al. it would mean dropping a package, and adopting a new one at the same time.

\section{Aggregation of coupling and effort}
