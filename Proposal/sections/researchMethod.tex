% !TEX root = ..\main.tex
\section{Research method} % Started
The main research method that is going to be employed during this project is \textit{Technical Action Research} (TAR) \cite{wieringa2012technical}.

This research method is artifact-based, which means that the first step is to produce the artifact meant to be used in certain situations envisioned by the researcher. The testing of this artifact, to see if it is effective in these situations, is done through a number of iterations. First, under ideal conditions, and then changing the experiments step by step to reach a real-world situation. These iterations, in the context of this project, are going to be limited due to the existing deadlines. However, there is the option of continuing with this part of the work in the future.

\bigskip\noindent
In addition, the research will also include controlled experiments. These experiments will be conducted as the empirical part of the validation of the coupling metrics. Additionally, the empirical validation of the effort estimation is going to be achieved by the means of case studies. In these case studies, the estimated effort will be compared with the real effort that the change required.

\bigskip\noindent
Some of the most difficult parts of this research are the validation and adaptation of the coupling metrics.

First, the validation since there is not a unique way to do it, but rather a wide variety of criteria and approaches. Therefore, conducting a complete validation is a complex task as it is not possible to validate every aspect of the metric.

In addition, even though the metrics that are going to be used to measure coupling between the projects and the libraries already exist, they are going to be adapted. These changes of the metrics could be complex for some of them. However, there are many metrics that could be used, and the ones that can not be adapted could serve as an indication of the types of dependencies that could exist when adapting other metrics.

\bigskip\noindent
After the formal definition and theoretical validation of the metrics for both, the coupling measurement and the estimation effort, to replace a library, a Proof-of-Concept (PoC) will be made. With this PoC, the metrics will be calculated for the projects given to the PoC. Once the PoC is ready, the empirical validation will be carried out for each of the metrics, by conducting controlled experiments and case studies as elaborated above.

\bigskip\noindent
The methodology for validating the metrics chosen during this research will be divided into two phases: a first phase of theoretical validation and a second one of empirical validation \cite{srinivasan2014software}.
These validations will be carried out during the phases \textit{Creation of the theoretical model} and \textit{Evaluation of the project} respectively (see Section \ref{section:timeline}).

\paragraph{Literature survey}
During the literature survey for this proposal we mainly utilized \textit{Google Scholar}. The resources found were documented with the following structure: First, by the overarching area of research, in this case: Dependency management, dependency/coupling metrics, effort estimation, and metrics validation.

As for the metrics validation category, we have conducted a brief excerpt of the main points or useful information extracted from each of the papers.
Finally, the papers of coupling metrics are used to extract the characteristics of each of the proposed metrics to filter out the non-applicable metrics (see Table \ref{table:coupling-metrics}). Besides, the effort estimation papers were used as a survey of the major models in order to compare them.

See section \ref{section:literature} for more detail.

\bigskip\noindent
Once the literature survey is finalized, we expect the following input:

\begin{itemize}
  \item A clear overview of the existing coupling metrics, as well as a comparison between them, in order to have the necessary knowledge to determine if a metric is applicable to the situation considered in this project, and how could it be adapted.

  \item The different existing models to estimate effort. An overview of the necessary steps and the data to estimate effort and the differences between the models.

  \item The various approaches to validate metrics, and the different characteristics of the methodologies. Also, the awareness of which methodologies and criteria could be applied to the proposed metrics.

  \item The existing options to create a call-level dependency graph for the PoC.
\end{itemize}

\paragraph{Timeline}
The timeline of this project will be divided in five main parts. A description of each one of these parts can be found in Section \ref{section:timeline}, as well as a timeline in the form of a Gantt diagram, in Figure \ref{figure:gantt-chart}.
