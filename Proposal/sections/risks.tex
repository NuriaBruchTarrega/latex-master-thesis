% !TEX root = ..\main.tex
\section{Risks} % Acabar i corrector
\paragraph{Project goals are too ambitious.}
If the project goals are underestimated, it could cause a problem in terms of project completion.
To minimize the possibility of this risk becoming reality, it is critical to clearly specify the goals of the project.
To be able to react if the risk is not prevented, some extra time has been not included in the planned timeline.
First, more hours could be invested as overtime and weeks 34 and 35 can also be added to the project, while the final deadline would still not be missed.

\paragraph{Project goals are too vague.}
This risk is also related to the definition of the project objectives. As with the previous risk, to prevent it, it is important to dedicate the necessary time and effort to specify the objectives.
However, if this situation is detected, the development of the project will be stopped to reassess and specify the objectives in more detail.
In this way, it will be avoided that this risk leads to major problems in later stages of the project.

\paragraph{The information about the project is unavailable/incomplete/too difficult.}
In order for this risk not to lead to severe problems during the project, it is important to gather information on the topic as soon as possible.
If it is noticed that there is not enough information or the project is overly complex, efforts will be focused on the part of the project that is most feasible.
In this way, the project would be continued; although the result would not be as complete as planned, it would serve as a starting point for future continuation.

\paragraph{The gap between the current expertise and the required expertise is too large.}
To mitigate the possibility that this risk becomes reality and results in a grave problem for the holistic completion of the project,
during the first phase of the project, time has been allocated for knowledge gathering. This is meant to reduce the gap between the required and current knowledge.
In the same way, it could be detected that it is necessary to invest more hours in the training phase.
These hours would be spent in the form of overtime, which would not affect the estimated date of completion of the project.

\paragraph{Adaptation and justification of the coupling metrics.}
During the second phase of the project, the coupling metrics are going to be adapted. One of the risks associated with this task is that the metrics may not be flexible enough to be used in a different context.
To prevent this, we selected several metrics that could be used. Therefore, if one of them is not feasible to adapt, we can use the rest. However, if this is the case, the characteristics of the discarded metric and the type of coupling that it measures would be considered when adapting another metric.

\paragraph{Adaptation of the effort to change code calculation.}
This risk is similar to the previous risk. In this case, we would have to consider an alternative model to estimate the effort to implement in the context studied in this project. To do so, it is important to consult the support from experts in the field of research, as well as the first phase of the project. In this initial phase, it is expected for us to acquire the necessary knowledge about the model to be able to assess if the adaptation is feasible.

\paragraph{Validation of the metrics.}
Since there is no unique and standardized way to validate metrics, this task entails a risk for the project. The validation will be performed for certain properties and validation criteria for each of the used metrics. However, the time limit does not allow for all of the metrics to be validated using each of the criteria, or conduct too many experiments. Therefore, once the project is finalized, we could affirm that the metrics have been validated, but it will not be a complete validation.
